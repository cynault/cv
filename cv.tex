%% cv.tex
%% Copyright 2017 Zeyi Fan
%
% This work may be distributed and/or modified under the
% conditions of the LaTeX Project Public License, either version 1.3
% of this license or (at your option) any later version.
% The latest version of this license is in
%   http://www.latex-project.org/lppl.txt
% and version 1.3 or later is part of all distributions of LaTeX
% version 2005/12/01 or later.
%
% This work has the LPPL maintenance status `maintained'.
%
% The Current Maintainer of this work is Zeyi Fan.
%
% This work consists of only the file cv.tex

\documentclass{ctexrep}
\usepackage[top=.4in, bottom=.0in, left=3in, right=0.4in,marginparwidth=2.5in]{geometry}
\usepackage{tikz}
\usepackage{xcolor}
\usepackage[absolute,overlay]{textpos}
\usepackage{fontspec}
\usepackage{titlesec}
\usepackage{pstricks}
\usepackage{amssymb}
\usepackage{paralist}

%\usepackage[pdfauthor={Zeyi Fan},
            %pdftitle={Zeyi Fan's Resume},
            %pdfkeywords={}]{hyperref}
\usepackage{hyperref}

\definecolor{mygray}{gray}{0.95}
\definecolor{lightdark}{gray}{0.55}
\definecolor{dark}{gray}{0.3}
\definecolor{skillbg}{gray}{0.7}

\newcommand{\amount}{5.7in}
\setcounter{section}{-1}

\linespread{1.2}
\pagenumbering{gobble}

\renewcommand{\labelitemi}{$\blacksquare$}
\renewenvironment{itemize}[1]{\begin{compactitem}#1}{\end{compactitem}}

% Helpers

\newcommand\twodigits[1]{%
   \ifnum#1<10 0#1\else #1\fi
}

%\newcommand\hl[1]{{\ralewaysb #1}}
\newcommand\hl[1]{#1}


% Fonts

\newfontfamily\raleway{Raleway}
\newfontfamily\ralewaym{Raleway Medium}
\newfontfamily\ralewaysb{Raleway SemiBold}
\newfontfamily\ralewayb{Raleway Bold}
\newfontfamily\ralewayeb{Raleway ExtraBold}
\newfontfamily\ralewaybb{Raleway Black}

\setmainfont{Raleway}


% Styles
\newcommand{\name}[2]{
    \begin{center}
        \Huge{
            \ralewayeb{#1 #2}
        }
    \end{center}
}

\newcommand{\tagline}[1]{
    \begin{center}
        \large{
            \color{dark}
            \ralewaysb{#1}
        }
        \vspace{.2em}
    \end{center}
}

\renewcommand{\thesection}{\twodigits{\arabic{section}}.}

\titleformat{\section}
[hang]
{\ralewaysb\large\color{dark}}
{}
{.0em}
{}
[{\titlerule[0.8pt]}]

\titlespacing*{\section}{0pt}{4pt}{10pt}

\setlength{\TPHorizModule}{1mm}
\setlength{\TPVertModule}{1mm}
\setlength{\parindent}{0mm}

\newcommand{\contactline}[2]{
    \ralewaysb{#1} & \raleway{#2}
}

\newcommand{\skill}[2]{
    \vspace{.2em}
    \ralewayb{#1}

    \psset{xunit=0.197\linewidth, yunit=5pt}
    \begin{pspicture}[showgrid=false](5,1)
        \psline[linecolor=skillbg](0,0.5)(5,0.5)
        \psline[linecolor=skillbg,arrows=|-|](1,0.5)(2,0.5)
        \psline[linecolor=skillbg,arrows=|-|](3,0.5)(4,0.5)
        \psline[linecolor=skillbg,arrows=-|](0,0.5)(5,0.5)

        \psline[linecolor=black,arrows=|-|](0,0.5)(#2,0.5)
    \end{pspicture}
}

\newcommand{\block}[3]{
    {\large\ralewayb #1}

    \ralewaym{\color{lightdark}#2}

    \raleway{#3}
}

\begin{document}

% background

\begin{tikzpicture}[remember picture,overlay]
  \fill[mygray] (current page.south west) rectangle ([xshift=-\amount]current page.north east);
\end{tikzpicture}

% Side Bar

\begin{textblock}{58.5}(6,7.7)
    \name{王若愚}

    %\tagline{Student \& Developer}

    \section{关于我}

    \color{dark}

    王若愚是广州大学的研二研究生,专业是网络空间安全,ctf pwn手,他对网络安全攻防充满热情,对待工作认真,重视实践操作。技术栈为二进制攻防、编程语言相关技术,熟悉iot的漏洞利用,对Android、Windows下的攻防有过实践操作,自学能力强。

    \section{联系方式}

    \renewcommand{\arraystretch}{1.1}

    \begin{tabular}{rl}
        \contactline{Phone}{(86) 13388281278} \\
        \contactline{Email}{cynault@163.com} \\
        

        
    \end{tabular}

    \section{技能}

    % 编程语言

    
    \skill{C/C++}{4}


    
    \skill{Python}{3}


    % 系统平台
    \skill{Linux}{4}

    \skill{windows}{2}

    \skill{Android}{3}
   
    % 工具
    \skill{gdb}{4}

    \skill{IDA Pro}{4}

    \skill{qemu}{2}

    \skill{pwntools}{3}

    \skill{Git}{3}

    
    \skill{BurpSuite}{3}

    \skill{ARM}{3}

    \skill{X86/64}{4}


\end{textblock}

% Main

\vspace{-25pt}  % Change this value to adjust the position of the right
\section{教育背景}

\block{广州大学}
{网络空间安全                                        \hfill 2021.09.01 - 至今}
{
    相关课程: 安全协议;网络安全综合实验;密码学。
}

\vspace{.5em}

\block{沈阳工业大学}
{智能科学与技术                             \hfill 2016.09.01 - 2020.06.01}
{
    相关课程: 神经网络;机器学习;自动控制原理。
}



\section{项目经历}

\block{物联网设备漏洞检测 \hfill 2023.2 - 2023.4}{}{ 
    \begin{itemize}[-]
        \item 物联网所内项目,对存在漏洞设备进行攻击利用。
        \item 对cisco设备(RV110、RV340)、小米设备(R3P)的漏洞进行分析评估。
        \item 对现有的poc进行分析学习复现(RV110、R3P)。
        \item 对已公布漏洞,但未有系统形成命令执行EXP的设备,进行分析,对其多个漏洞进行系统梳理,构造最终命令执行python脚本(RV340)。
    \end{itemize}
}




\vspace{.5em}

\block{基于管道进程迁移项目 \hfill 2023.1 - 2023.2}{}{ 
    \begin{itemize}[-]
        \item 白帽社区红队持久化权限维持工具。
        \item 目标是实现进程注入后,目标进程挂掉后能进行迁移,自动进行注入。
        \item 基于C++语言,进程注入技术,以及管道技术。
        \item 涉及进程通信,设计一套进程间主服务端灵活转换的方案。
    \end{itemize}
}

\vspace{.5em}



\block{蜜罐项目 \hfill 2022.12}{}{ 
    \begin{itemize}[-]
        \item datacon的网络流量检测项目
        \item 使用C\#语言,asp.net core框架,规则匹配,实现了对漏扫工具扫描结果的混淆。
        \item 抓取漏扫工具的攻击流量,分析出大部分流量来自于xray。
        \item 对poc进行分析,其验证漏洞的方式,如命令执行验证,关键字验证,存储型验证。
    \end{itemize}
}






\vspace{.5em}


\block{小米设备协议安全评估 {\hfill 2022.4 - 2022.6}}{}{
    \begin{itemize}[-]
        \item 安全协议实验项目
        \item 发现小米设备存在协议漏洞。
        \item 小米S1台灯作为目标设备,BurpSuite抓包,分析手机APP米家、台灯以及云服务器的通信过程。
        \item 认证过程存在缺陷,设备id泄露后,攻击者拥有控制权。
        
    \end{itemize}
}

\section{获奖情况}

\ralewaysb 三等奖 \raleway 第一届中国研究生网络安全创新大赛\hfill 2022.11 

\ralewaysb 二等奖 \raleway 第十五届全国大学生信息安全竞赛华南赛区\hfill 2022.9

\ralewaysb 三等奖 \raleway CICV智能网联汽车漏洞挖掘赛(线上、线下)\hfill 2022.5 、2022.8

\section{教学服务}

\ralewaysb 讲师 \raleway ctf-pwn攻与防,无锡税务局\hfill 2022.8

\ralewaysb 助教 \raleway  计算机网络,广州大学  \hfill 2022.3 - 2022.7 

\end{document}
